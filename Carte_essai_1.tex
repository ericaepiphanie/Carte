\documentclass[]{article}
\usepackage{lmodern}
\usepackage{amssymb,amsmath}
\usepackage{ifxetex,ifluatex}
\usepackage{fixltx2e} % provides \textsubscript
\ifnum 0\ifxetex 1\fi\ifluatex 1\fi=0 % if pdftex
  \usepackage[T1]{fontenc}
  \usepackage[utf8]{inputenc}
\else % if luatex or xelatex
  \ifxetex
    \usepackage{mathspec}
  \else
    \usepackage{fontspec}
  \fi
  \defaultfontfeatures{Ligatures=TeX,Scale=MatchLowercase}
\fi
% use upquote if available, for straight quotes in verbatim environments
\IfFileExists{upquote.sty}{\usepackage{upquote}}{}
% use microtype if available
\IfFileExists{microtype.sty}{%
\usepackage{microtype}
\UseMicrotypeSet[protrusion]{basicmath} % disable protrusion for tt fonts
}{}
\usepackage[margin=1in]{geometry}
\usepackage{hyperref}
\hypersetup{unicode=true,
            pdftitle={Carte\_BF},
            pdfborder={0 0 0},
            breaklinks=true}
\urlstyle{same}  % don't use monospace font for urls
\usepackage{color}
\usepackage{fancyvrb}
\newcommand{\VerbBar}{|}
\newcommand{\VERB}{\Verb[commandchars=\\\{\}]}
\DefineVerbatimEnvironment{Highlighting}{Verbatim}{commandchars=\\\{\}}
% Add ',fontsize=\small' for more characters per line
\usepackage{framed}
\definecolor{shadecolor}{RGB}{248,248,248}
\newenvironment{Shaded}{\begin{snugshade}}{\end{snugshade}}
\newcommand{\AlertTok}[1]{\textcolor[rgb]{0.94,0.16,0.16}{#1}}
\newcommand{\AnnotationTok}[1]{\textcolor[rgb]{0.56,0.35,0.01}{\textbf{\textit{#1}}}}
\newcommand{\AttributeTok}[1]{\textcolor[rgb]{0.77,0.63,0.00}{#1}}
\newcommand{\BaseNTok}[1]{\textcolor[rgb]{0.00,0.00,0.81}{#1}}
\newcommand{\BuiltInTok}[1]{#1}
\newcommand{\CharTok}[1]{\textcolor[rgb]{0.31,0.60,0.02}{#1}}
\newcommand{\CommentTok}[1]{\textcolor[rgb]{0.56,0.35,0.01}{\textit{#1}}}
\newcommand{\CommentVarTok}[1]{\textcolor[rgb]{0.56,0.35,0.01}{\textbf{\textit{#1}}}}
\newcommand{\ConstantTok}[1]{\textcolor[rgb]{0.00,0.00,0.00}{#1}}
\newcommand{\ControlFlowTok}[1]{\textcolor[rgb]{0.13,0.29,0.53}{\textbf{#1}}}
\newcommand{\DataTypeTok}[1]{\textcolor[rgb]{0.13,0.29,0.53}{#1}}
\newcommand{\DecValTok}[1]{\textcolor[rgb]{0.00,0.00,0.81}{#1}}
\newcommand{\DocumentationTok}[1]{\textcolor[rgb]{0.56,0.35,0.01}{\textbf{\textit{#1}}}}
\newcommand{\ErrorTok}[1]{\textcolor[rgb]{0.64,0.00,0.00}{\textbf{#1}}}
\newcommand{\ExtensionTok}[1]{#1}
\newcommand{\FloatTok}[1]{\textcolor[rgb]{0.00,0.00,0.81}{#1}}
\newcommand{\FunctionTok}[1]{\textcolor[rgb]{0.00,0.00,0.00}{#1}}
\newcommand{\ImportTok}[1]{#1}
\newcommand{\InformationTok}[1]{\textcolor[rgb]{0.56,0.35,0.01}{\textbf{\textit{#1}}}}
\newcommand{\KeywordTok}[1]{\textcolor[rgb]{0.13,0.29,0.53}{\textbf{#1}}}
\newcommand{\NormalTok}[1]{#1}
\newcommand{\OperatorTok}[1]{\textcolor[rgb]{0.81,0.36,0.00}{\textbf{#1}}}
\newcommand{\OtherTok}[1]{\textcolor[rgb]{0.56,0.35,0.01}{#1}}
\newcommand{\PreprocessorTok}[1]{\textcolor[rgb]{0.56,0.35,0.01}{\textit{#1}}}
\newcommand{\RegionMarkerTok}[1]{#1}
\newcommand{\SpecialCharTok}[1]{\textcolor[rgb]{0.00,0.00,0.00}{#1}}
\newcommand{\SpecialStringTok}[1]{\textcolor[rgb]{0.31,0.60,0.02}{#1}}
\newcommand{\StringTok}[1]{\textcolor[rgb]{0.31,0.60,0.02}{#1}}
\newcommand{\VariableTok}[1]{\textcolor[rgb]{0.00,0.00,0.00}{#1}}
\newcommand{\VerbatimStringTok}[1]{\textcolor[rgb]{0.31,0.60,0.02}{#1}}
\newcommand{\WarningTok}[1]{\textcolor[rgb]{0.56,0.35,0.01}{\textbf{\textit{#1}}}}
\usepackage{graphicx,grffile}
\makeatletter
\def\maxwidth{\ifdim\Gin@nat@width>\linewidth\linewidth\else\Gin@nat@width\fi}
\def\maxheight{\ifdim\Gin@nat@height>\textheight\textheight\else\Gin@nat@height\fi}
\makeatother
% Scale images if necessary, so that they will not overflow the page
% margins by default, and it is still possible to overwrite the defaults
% using explicit options in \includegraphics[width, height, ...]{}
\setkeys{Gin}{width=\maxwidth,height=\maxheight,keepaspectratio}
\IfFileExists{parskip.sty}{%
\usepackage{parskip}
}{% else
\setlength{\parindent}{0pt}
\setlength{\parskip}{6pt plus 2pt minus 1pt}
}
\setlength{\emergencystretch}{3em}  % prevent overfull lines
\providecommand{\tightlist}{%
  \setlength{\itemsep}{0pt}\setlength{\parskip}{0pt}}
\setcounter{secnumdepth}{0}
% Redefines (sub)paragraphs to behave more like sections
\ifx\paragraph\undefined\else
\let\oldparagraph\paragraph
\renewcommand{\paragraph}[1]{\oldparagraph{#1}\mbox{}}
\fi
\ifx\subparagraph\undefined\else
\let\oldsubparagraph\subparagraph
\renewcommand{\subparagraph}[1]{\oldsubparagraph{#1}\mbox{}}
\fi

%%% Use protect on footnotes to avoid problems with footnotes in titles
\let\rmarkdownfootnote\footnote%
\def\footnote{\protect\rmarkdownfootnote}

%%% Change title format to be more compact
\usepackage{titling}

% Create subtitle command for use in maketitle
\providecommand{\subtitle}[1]{
  \posttitle{
    \begin{center}\large#1\end{center}
    }
}

\setlength{\droptitle}{-2em}

  \title{Carte\_BF}
    \pretitle{\vspace{\droptitle}\centering\huge}
  \posttitle{\par}
    \author{}
    \preauthor{}\postauthor{}
    \date{}
    \predate{}\postdate{}
  

\begin{document}
\maketitle

\begin{Shaded}
\begin{Highlighting}[]
\CommentTok{#Importation du package}
\KeywordTok{library}\NormalTok{(raster)}
\end{Highlighting}
\end{Shaded}

\begin{verbatim}
## Warning: package 'raster' was built under R version 3.6.3
\end{verbatim}

\begin{verbatim}
## Loading required package: sp
\end{verbatim}

\begin{verbatim}
## Warning: package 'sp' was built under R version 3.6.3
\end{verbatim}

\begin{Shaded}
\begin{Highlighting}[]
\CommentTok{#Ensemble de la Burkina}

\NormalTok{BF_Formes <-}\StringTok{ }\KeywordTok{getData}\NormalTok{(}\DataTypeTok{name=}\StringTok{"GADM"}\NormalTok{, }\DataTypeTok{country=}\StringTok{"BFA"}\NormalTok{, }\DataTypeTok{level=}\DecValTok{1}\NormalTok{)}
\NormalTok{colors=}\KeywordTok{rainbow}\NormalTok{(}\KeywordTok{length}\NormalTok{(BF_Formes}\OperatorTok{$}\NormalTok{NAME_}\DecValTok{1}\NormalTok{))}
\KeywordTok{plot}\NormalTok{(BF_Formes, }\DataTypeTok{col=}\NormalTok{colors, }\DataTypeTok{main=}\StringTok{"Carte du Burkina Faso"}\NormalTok{)}
\KeywordTok{text}\NormalTok{(}\KeywordTok{coordinates}\NormalTok{(BF_Formes), }\DataTypeTok{labels =}\NormalTok{ BF_Formes}\OperatorTok{$}\NormalTok{NAME_}\DecValTok{1}\NormalTok{,}\DataTypeTok{cex=}\FloatTok{0.6}\NormalTok{)}
\end{Highlighting}
\end{Shaded}

\includegraphics{Carte_essai_1_files/figure-latex/unnamed-chunk-1-1.pdf}

\begin{Shaded}
\begin{Highlighting}[]
\CommentTok{#legend("topleft",legend=BF_Formes$NAME_1,fill=colors,cex=1.3,bty="n" )}
\end{Highlighting}
\end{Shaded}

\begin{Shaded}
\begin{Highlighting}[]
\CommentTok{#Importation du package}
\KeywordTok{library}\NormalTok{(raster)}
\CommentTok{#Importation d'une base sur les ménages vulnérables}
\KeywordTok{library}\NormalTok{(readxl)}
\NormalTok{Nb_menage <-}\StringTok{ }\KeywordTok{read_excel}\NormalTok{(}\StringTok{"C:/Users/USER/Desktop/Carte_BF avec R/Nb_menage.xlsx"}\NormalTok{)}


\CommentTok{#Obtenir les formes}

\NormalTok{BF_Formes <-}\StringTok{ }\KeywordTok{getData}\NormalTok{(}\DataTypeTok{name=}\StringTok{"GADM"}\NormalTok{, }\DataTypeTok{country=}\StringTok{"BFA"}\NormalTok{, }\DataTypeTok{level=}\DecValTok{3}\NormalTok{)}
\KeywordTok{plot}\NormalTok{(BF_Formes, }\DataTypeTok{main=}\StringTok{"Carte du Burkina Faso"}\NormalTok{)}
\end{Highlighting}
\end{Shaded}

\includegraphics{Carte_essai_1_files/figure-latex/unnamed-chunk-4-1.pdf}

\begin{Shaded}
\begin{Highlighting}[]
\KeywordTok{names}\NormalTok{(BF_Formes)}
\end{Highlighting}
\end{Shaded}

\begin{verbatim}
##  [1] "GID_0"     "NAME_0"    "GID_1"     "NAME_1"    "NL_NAME_1"
##  [6] "GID_2"     "NAME_2"    "NL_NAME_2" "GID_3"     "NAME_3"   
## [11] "VARNAME_3" "NL_NAME_3" "TYPE_3"    "ENGTYPE_3" "CC_3"     
## [16] "HASC_3"
\end{verbatim}

\begin{Shaded}
\begin{Highlighting}[]
\CommentTok{#Établissement de l'index}
\NormalTok{idx <-}\StringTok{ }\KeywordTok{match}\NormalTok{(BF_Formes}\OperatorTok{$}\NormalTok{NAME_}\DecValTok{3}\NormalTok{, }\KeywordTok{as.character}\NormalTok{(Nb_menage}\OperatorTok{$}\NormalTok{Name))}

\CommentTok{#Tranfert des données pour une variable (le nombre de votants sur le nombre d'inscrits) en fonction de la règle de concordance}
\NormalTok{concordance <-}\StringTok{ }\NormalTok{Nb_menage[idx, }\StringTok{"Nombre_menage"}\NormalTok{]}
\NormalTok{BF_Formes}\OperatorTok{$}\NormalTok{Nombre_menage <-}\StringTok{ }\NormalTok{concordance}

\CommentTok{#Tracage de la carte}
\CommentTok{#établissemment de la charte des coupeurs puis tracage de la carte en utilisant}
\NormalTok{couleurs <-}\StringTok{ }\KeywordTok{rainbow}\NormalTok{(}\DecValTok{99}\NormalTok{, }\DataTypeTok{start=}\NormalTok{.}\DecValTok{1}\NormalTok{)}
\CommentTok{#spplot(BF_Formes, "Nombre_menage", col.regions= couleurs,  main=list(label="Nombre de ménage BF",cex=1.3))}
\KeywordTok{spplot}\NormalTok{(BF_Formes, }\StringTok{"Nombre_menage"}\NormalTok{, }\DataTypeTok{palette=}\StringTok{"Greens"}\NormalTok{,  }\DataTypeTok{main=}\KeywordTok{list}\NormalTok{(}\DataTypeTok{label=}\StringTok{"Nombre de ménage BF"}\NormalTok{,}\DataTypeTok{cex=}\FloatTok{1.3}\NormalTok{))}
\end{Highlighting}
\end{Shaded}

\includegraphics{Carte_essai_1_files/figure-latex/unnamed-chunk-4-2.pdf}

\begin{Shaded}
\begin{Highlighting}[]
\CommentTok{#ccCarte12}
\end{Highlighting}
\end{Shaded}


\end{document}
